\documentclass[a4paper,12pt]{extarticle}

% ============= FONT AND TEXT PACKAGES =============
\usepackage[T1]{fontenc}
\usepackage{charter}  % Beautiful serif font
\usepackage[expert]{mathdesign}  % Matching math fonts
\usepackage{cite}  % Citation management

% ============= MATH PACKAGES =============
\usepackage{amsmath,amssymb,amsfonts}
\usepackage{algorithmic}

% ============= GRAPHICS AND COLOR PACKAGES =============
\usepackage{graphicx}
\usepackage{textcomp}
\usepackage[table,dvipsnames]{xcolor}  % Enhanced color support
\usepackage{tikz}
\usetikzlibrary{shapes.geometric, arrows, shadows, positioning, decorations.pathmorphing, calc}
\usepackage{tcolorbox}  % Colored boxes
\tcbuselibrary{skins,breakable}

% ============= TABLE PACKAGES =============
\usepackage{booktabs}
\usepackage{float}
\usepackage{makecell}
\usepackage{multirow}
\usepackage{tabularx}
\usepackage{longtable}
\usepackage{array}

% ============= LAYOUT PACKAGES =============
\usepackage[top=2.5cm, bottom=2.5cm, left=2.5cm, right=2.5cm]{geometry}
\usepackage{setspace}
\usepackage{multicol}
\usepackage[absolute,overlay]{textpos}
\usepackage{subcaption}
\usepackage{fancyhdr}  % Custom headers and footers
\usepackage{titlesec}  % Custom section titles

% ============= HYPERREF AND URL =============
\usepackage{url}
\usepackage[colorlinks=true, linkcolor=NavyBlue, citecolor=OliveGreen, urlcolor=RoyalBlue]{hyperref}

% ============= COLOR DEFINITIONS =============
\definecolor{maincolor}{RGB}{0, 51, 102}  % Deep blue
\definecolor{accentcolor}{RGB}{220, 50, 50}  % Vibrant red
\definecolor{sectioncolor}{RGB}{0, 102, 204}  % Bright blue
\definecolor{subsectioncolor}{RGB}{51, 102, 153}  % Medium blue
\definecolor{lightgray}{RGB}{240, 240, 240}  % Light background
\definecolor{darkgray}{RGB}{100, 100, 100}  % Dark text
\definecolor{codebackground}{RGB}{248, 248, 255}  % Code background

% ============= CUSTOM SECTION FORMATTING =============
\titleformat{\section}
  {\normalfont\LARGE\bfseries\color{sectioncolor}}
  {\thesection}{1em}{}
  [\titlerule]

\titleformat{\subsection}
  {\normalfont\Large\bfseries\color{subsectioncolor}}
  {\thesubsection}{1em}{}

\titleformat{\subsubsection}
  {\normalfont\large\bfseries\color{darkgray}}
  {\thesubsubsection}{1em}{}

% Custom section formatting for unnumbered sections
\titleformat{name=\section,numberless}
  {\normalfont\LARGE\bfseries\color{sectioncolor}}
  {}{0em}{}
  [\titlerule]

\titleformat{name=\subsection,numberless}
  {\normalfont\Large\bfseries\color{subsectioncolor}}
  {}{0em}{}

% ============= FANCY HEADERS =============
\pagestyle{fancy}
\fancyhf{}
\fancyhead[L]{\small\color{maincolor}\leftmark}
\fancyhead[R]{\small\color{maincolor}OMNeT++ Email Security Simulation}
\fancyfoot[C]{\small\color{maincolor}\thepage}
% Increase headheight to satisfy fancyhdr
\setlength{\headheight}{24.2pt}
\renewcommand{\headrulewidth}{0.5pt}
\renewcommand{\footrulewidth}{0.5pt}
\renewcommand{\headrule}{\hbox to\headwidth{\color{maincolor}\leaders\hrule height \headrulewidth\hfill}}
\renewcommand{\footrule}{\hbox to\headwidth{\color{maincolor}\leaders\hrule height \footrulewidth\hfill}}

% ============= TABLE SETTINGS =============
\setlength{\arrayrulewidth}{0.5pt}
\setlength{\tabcolsep}{10pt}
\renewcommand{\arraystretch}{1.3}
\newcolumntype{s}{>{} p{0.5cm}}
% Avoid shadowing array's built-in m{} column type
\newcolumntype{Y}{>{} X}

% ============= LIST SETTINGS =============
\usepackage{enumitem}
\setlist{nosep, leftmargin=*}
\setlist[itemize]{nosep, leftmargin=1.5em, topsep=0pt, partopsep=0pt, parsep=0pt, itemsep=3pt, label=$\triangleright$}
\setlist[enumerate]{nosep, leftmargin=1.5em, topsep=0pt, partopsep=0pt, parsep=0pt, itemsep=3pt}

% ============= CUSTOM BOXES =============
% Info box for technical details
\newtcolorbox{infobox}[1][]{
  colback=blue!5!white,
  colframe=blue!75!black,
  fonttitle=\bfseries,
  title=#1,
  enhanced,
  attach boxed title to top left={yshift=-2mm, xshift=5mm},
  boxed title style={colback=blue!75!black}
}

% Warning box for security notes
\newtcolorbox{warningbox}[1][]{
  colback=red!5!white,
  colframe=red!75!black,
  fonttitle=\bfseries,
  title=#1,
  enhanced,
  attach boxed title to top left={yshift=-2mm, xshift=5mm},
  boxed title style={colback=red!75!black}
}

% Success box for features
\newtcolorbox{successbox}[1][]{
  colback=green!5!white,
  colframe=green!65!black,
  fonttitle=\bfseries,
  title=#1,
  enhanced,
  attach boxed title to top left={yshift=-2mm, xshift=5mm},
  boxed title style={colback=green!65!black}
}

% Module box for module descriptions
\newtcolorbox{modulebox}[2][]{
  colback=lightgray,
  colframe=maincolor,
  fonttitle=\bfseries\color{white},
  title=#2,
  enhanced,
  rounded corners,
  boxrule=1pt,
  left=5pt, right=5pt, top=5pt, bottom=5pt,
  attach boxed title to top center={yshift=-3mm},
  boxed title style={colback=maincolor, rounded corners}
}

% ============= CUSTOM COMMANDS =============
\newcommand{\bfheading}[1]{\noindent\textbf{\color{subsectioncolor}#1}}
\newcommand{\code}[1]{\texttt{\colorbox{codebackground}{#1}}}
\newcommand{\modulename}[1]{\textbf{\color{maincolor}#1}}
\newcommand{\protocol}[1]{\textbf{\color{accentcolor}#1}}

% ============= SPACING =============
\setlength{\parskip}{0.5em}
\setlength{\parindent}{0pt}


\begin{document}

% Set page numbering to Roman numerals for front matter
\pagenumbering{roman}
\pagestyle{empty}  % No header/footer on title page

% ===================== MODERN TITLE PAGE =====================
\begin{titlepage}
    \begin{tikzpicture}[remember picture,overlay]
        % Background gradient effect
        \fill[maincolor] (current page.north west) rectangle ([yshift=-8cm]current page.north east);
        \fill[white] ([yshift=-8cm]current page.north west) rectangle (current page.south east);
        
        % Decorative elements
        \foreach \i in {0,0.5,...,8} {
            \draw[white, opacity=0.1, line width=1pt] ([yshift=-\i cm]current page.north west) -- ([yshift=-\i cm]current page.north east);
        }
    \end{tikzpicture}
    
    \centering
    
    % Course Information
    \begin{tcolorbox}[
        colback=white,
        colframe=maincolor,
        width=0.85\textwidth,
        arc=3mm,
        boxrule=2pt,
        halign=center,
        valign=center,
        left=5pt, right=5pt, top=8pt, bottom=8pt
    ]
        {\fontsize{14}{17}\selectfont \textbf{\color{maincolor}\textsc{CSE 4106 : Computer Netowrks Laboratory }}}
    \end{tcolorbox}

    \vspace{0.5cm}
    
    % Title with modern styling
    \begin{tcolorbox}[
        enhanced,
        colback=accentcolor!5,
        colframe=accentcolor,
        width=0.9\textwidth,
        arc=5mm,
        boxrule=2pt,
        drop shadow={opacity=0.3},
        halign=center,
        valign=center,
        left=10pt, right=10pt, top=15pt, bottom=15pt
    ]
        \begin{spacing}{1.3}
        {\fontsize{20}{24}\selectfont \textbf{\color{maincolor}
        Report on Secure Email \\Communication Simulation:\\[0.3cm]
        \large Man-in-the-Middle Attack Analysis\\
        and Certificate-Based Defense}}
        \end{spacing}
    \end{tcolorbox}
    
    \vspace{0.1cm}
    
    % Author Information
    \begin{tcolorbox}[
        colback=lightgray,
        colframe=maincolor,
        width=0.6\textwidth,
        arc=3mm,
        boxrule=1.5pt,
        halign=center,
        left=5pt, right=5pt, top=10pt, bottom=10pt
    ]
        {\fontsize{13}{16}\selectfont \textbf{\color{darkgray}Submitted by}}\\[0.3cm]
        {\fontsize{16}{19}\selectfont \textbf{\color{maincolor}Md Sakibur Rahman}}\\[0.2cm]
        {\fontsize{12}{15}\selectfont \color{darkgray}Roll : 2007007}
    \end{tcolorbox}
    
    \vspace{0.5cm}
    
    % University Logo
\begin{center}
    \includegraphics[width=3.2cm, height=3.5cm]{logo.png}
\end{center}

    
    \vspace{0.8cm}
    
    % Supervisor Information
    \begin{tcolorbox}[
        enhanced,
        colback=white,
        colframe=sectioncolor,
        width=0.85\textwidth,
        arc=3mm,
        boxrule=1.5pt,
        halign=center,
        left=10pt, right=10pt, top=12pt, bottom=12pt
    ]
        {\fontsize{13}{16}\selectfont \textbf{\color{maincolor}Submitted To:}}\\[0.4cm]
        
        \begin{minipage}{0.45\textwidth}
            \centering
            {\fontsize{14}{17}\selectfont \textbf{\color{sectioncolor}Md Sakhawat Hossain}}\\[0.2cm]
            {\fontsize{11}{14}\selectfont \color{darkgray}Lecturer}\\
            {\fontsize{10}{13}\selectfont \color{darkgray}Dept. of Computer Science \& Engineering}
        \end{minipage}
        \hfill
        \begin{minipage}{0.45\textwidth}
            \centering
            {\fontsize{14}{17}\selectfont \textbf{\color{sectioncolor}Farhan Sadaf}}\\[0.2cm]
            {\fontsize{11}{14}\selectfont \color{darkgray}Lecturer}\\
            {\fontsize{10}{13}\selectfont \color{darkgray}Dept. of Computer Science \& Engineering}
        \end{minipage}
    \end{tcolorbox}
    
    \vfill
    
    % Bottom information
    \begin{tcolorbox}[
        colback=maincolor,
        colframe=maincolor,
        width=\textwidth,
        arc=0mm,
        boxrule=0pt,
        halign=center,
        left=5pt, right=5pt, top=10pt, bottom=10pt
    ]
        {\fontsize{12}{15}\selectfont \color{white}\textbf{Department of Computer Science and Engineering}}\\
        {\fontsize{12}{15}\selectfont \color{white}\textbf{Khulna University of Engineering \& Technology}}\\
        {\fontsize{11}{14}\selectfont \color{white}Khulna-9203, Bangladesh}\\[0.2cm]
    \end{tcolorbox}
    
\end{titlepage}

\newpage
\pagestyle{fancy}  % Restore header/footer

% Source Table



% Table of contents
\tableofcontents

\newpage

% List of figures
\listoffigures


% List of tables

\newpage

% Switch to Arabic page numbering starting from 1
\pagenumbering{arabic}

% ===================== OBJECTIVES SECTION =====================
\section{Project Objectives}

This project demonstrates vulnerabilities and defense mechanisms in secure email communication through Man-in-the-Middle (MITM) attack simulation using OMNeT++.

\textbf{Key Objectives:}
\begin{itemize}[label={\color{accentcolor}$\blacktriangleright$}]
    \item Implement SMTP server with IMAP and HTTP protocols
    \item Integrate Diffie-Hellman key exchange and RSA encryption
    \item Simulate and analyze MITM attacks on encrypted communication
    \item Demonstrate certificate-based authentication as defense mechanism
\end{itemize}


\section{Introduction}

This project explores critical vulnerabilities in encrypted communication through implementing a \textbf{Man-in-the-Middle (MITM)} attack on an email system. Built using \textbf{OMNeT++}, the simulation demonstrates how Diffie-Hellman key exchange and RSA encryption can be compromised without proper authentication.

\textbf{Attack Scenario:} A malicious sniffer positioned between sender (\code{mta\_Client\_SS}) and receiver (\code{mta\_Server\_RS}) intercepts communication. During Diffie-Hellman handshake, the attacker substitutes public keys to establish two separate encrypted sessions, enabling message decryption and modification.

\textbf{Defense Mechanism:} Digital certificates issued by a Certificate Authority (CA) prevent MITM attacks. The sniffer cannot forge valid certificate signatures without the CA's private key, forcing it into transparent mode where it can only forward unmodified messages.

\textbf{Key Insights:}
\begin{itemize}[label={\color{maincolor}$\star$}]
    \item Encryption alone is insufficient—authentication through certificates is crucial
    \item Dual-session architecture enables message interception and manipulation
    \item Certificate-based authentication provides robust protection against impersonation
\end{itemize}


\section{Theoretical Background}

\subsection{Man-in-the-Middle (MITM) Attack}

\subsubsection*{Attack Concept}

A \textbf{Man-in-the-Middle (MITM) attack} occurs when an attacker (Sniffer) secretly intercepts and controls communication between two parties (Sender and Receiver) who believe they are communicating directly with each other. The attacker positions themselves between the communicating parties, creating two separate encrypted sessions while remaining undetected.

\subsubsection*{Mathematical Analysis of MITM on Diffie-Hellman}

\textbf{Normal Diffie-Hellman Key Exchange (Without Attack):}

\begin{tcolorbox}[
    enhanced,
    colback=green!5,
    colframe=green!70!black,
    boxrule=1.5pt,
    arc=3mm,
    left=10pt, right=10pt, top=10pt, bottom=10pt
]
\textbf{Sender and Receiver establish shared secret:}
\begin{enumerate}
    \item Public parameters: generator $g$ and prime $p$
    \item Sender chooses private key $a$, computes public key: $A = g^a \bmod p$
    \item Receiver chooses private key $b$, computes public key: $B = g^b \bmod p$
    \item They exchange public keys: $A \leftrightarrow B$
    \item Sender computes: $k_{SR} = B^a \bmod p = (g^b)^a \bmod p = g^{ab} \bmod p$
    \item Receiver computes: $k_{SR} = A^b \bmod p = (g^a)^b \bmod p = g^{ab} \bmod p$
    \item \textbf{Result:} Both share the same secret key $k_{SR} = g^{ab} \bmod p$
\end{enumerate}
\end{tcolorbox}

\vspace{0.3cm}

\textbf{MITM Attack on Diffie-Hellman (Sniffer Intercepts):}

\begin{tcolorbox}[
    enhanced,
    colback=red!5,
    colframe=red!70!black,
    boxrule=1.5pt,
    arc=3mm,
    left=10pt, right=10pt, top=10pt, bottom=10pt
]
\textbf{Sniffer performs two separate DH key exchanges:}

\textbf{Phase 1: Key Substitution}
\begin{itemize}
    \item Sender computes $A = g^a \bmod p$ and sends to Receiver
    \item \textcolor{red}{\textbf{Sniffer intercepts}}, generates own key: $s$ (private), $S = g^s \bmod p$ (public)
    \item \textcolor{red}{\textbf{Sniffer substitutes}} $A$ with $S'$, sends $S' = g^s \bmod p$ to Receiver
    \item Receiver computes $B = g^b \bmod p$ and sends to Sender
    \item \textcolor{red}{\textbf{Sniffer intercepts}}, substitutes $B$ with $S''$, sends $S'' = g^s \bmod p$ to Sender
\end{itemize}

\textbf{Phase 2: Dual Session Establishment}
\begin{align*}
\text{\textbf{Sender computes:}} \quad & k_{SS} = (S'')^a \bmod p = (g^s)^a \bmod p = g^{sa} \bmod p \\
\text{\textbf{Receiver computes:}} \quad & k_{SR} = (S')^b \bmod p = (g^s)^b \bmod p = g^{sb} \bmod p \\
\text{\textbf{Sniffer computes:}} \quad & k_{SS} = A^s \bmod p = (g^a)^s \bmod p = g^{as} \bmod p \\
& k_{SR} = B^s \bmod p = (g^b)^s \bmod p = g^{bs} \bmod p
\end{align*}

\textbf{Phase 3: Message Relay}
\begin{itemize}
    \item Sender encrypts: $C_1 = E_{k_{SS}}(M)$ using key $g^{as} \bmod p$
    \item Sniffer decrypts: $M = D_{k_{SS}}(C_1)$ \textcolor{red}{\textbf{(reads plaintext!)}}
    \item Sniffer re-encrypts: $C_2 = E_{k_{SR}}(M)$ using key $g^{bs} \bmod p$
    \item Receiver decrypts: $M = D_{k_{SR}}(C_2)$ successfully
\end{itemize}

\textbf{Why the attack works:}
\begin{itemize}
    \item Sniffer computes $k_{SS} = A^s = (g^a)^s = g^{as} \bmod p$ (shared with Sender)
    \item Sniffer computes $k_{SR} = B^s = (g^b)^s = g^{bs} \bmod p$ (shared with Receiver)
    \item Sender and Receiver use \textbf{different keys}, both known to Sniffer
    \item \textcolor{red}{\textbf{Critical vulnerability:}} No authentication - cannot verify key ownership
\end{itemize}
\end{tcolorbox}

\textit{Reference: Chapter 19 of Understanding Cryptography by Christof Paar and Jan Pelzl}

\vspace{0.5cm}

\subsection{Certificate-Based Defense with RSA}

\subsubsection*{RSA Cryptosystem}

\textbf{RSA (Rivest-Shamir-Adleman)} is an asymmetric cryptographic algorithm that provides the mathematical foundation for digital signatures and certificates.

\begin{tcolorbox}[
    enhanced,
    colback=blue!5,
    colframe=blue!70!black,
    boxrule=1.5pt,
    arc=3mm,
    left=10pt, right=10pt, top=10pt, bottom=10pt
]
\textbf{RSA Key Generation:}
\begin{enumerate}
    \item Choose two large prime numbers: $p$ and $q$ (e.g., $p=61$, $q=53$)
    \item Compute modulus: $n = p \times q = 61 \times 53 = 3233$
    \item Compute Euler's totient: $\phi(n) = (p-1)(q-1) = 60 \times 52 = 3120$
    \item Choose public exponent: $e$ (typically 17 or 65537), where $\gcd(e, \phi(n)) = 1$
    \item Compute private exponent: $d \equiv e^{-1} \pmod{\phi(n)}$ (e.g., $d = 2753$)
    \item \textbf{Public key:} $(e, n) = (17, 3233)$
    \item \textbf{Private key:} $(d, n) = (2753, 3233)$
\end{enumerate}

\textbf{RSA Encryption/Decryption:}
\begin{align*}
\text{Encryption:} \quad & C = M^e \bmod n \\
\text{Decryption:} \quad & M = C^d \bmod n
\end{align*}

\textbf{Mathematical Property:}
\begin{equation*}
(M^e)^d \equiv M^{ed} \equiv M^{1} \equiv M \pmod{n}
\end{equation*}
This works because $ed \equiv 1 \pmod{\phi(n)}$ by construction.
\end{tcolorbox}

\vspace{0.3cm}



\subsubsection*{How Certificates Prevent MITM Attacks}

\begin{tcolorbox}[
    enhanced,
    colback=yellow!10,
    colframe=orange!80!black,
    boxrule=1.5pt,
    arc=3mm,
    left=10pt, right=10pt, top=10pt, bottom=10pt
]
\textbf{Certificate-Based Authentication Process:}

\textbf{Step 1: Certificate Issuance}
\begin{itemize}
    \item Sender requests certificate from CA
    \item CA verifies Sender's identity (out-of-band)
    \item CA creates certificate with Sender's public key
    \item CA signs: $\sigma_S = \text{hash}(\text{Sender\_Cert})^{d_{CA}} \bmod n_{CA}$
    \item CA sends $(\text{Sender\_Cert}, \sigma_S)$ to Sender
    \item Same process for Receiver: CA sends $(\text{Receiver\_Cert}, \sigma_R)$
\end{itemize}

\textbf{Step 2: Authenticated Key Exchange}
\begin{enumerate}
    \item Sender $\rightarrow$ $\{A = g^a \bmod p, \text{Sender\_Cert}, \sigma_S\}$ $\rightarrow$ Receiver
    \item Receiver verifies: $\text{hash}(\text{Sender\_Cert}) \stackrel{?}{=} \sigma_S^{e_{CA}} \bmod n_{CA}$
    \item If valid: Receiver trusts $A$ belongs to Sender
    \item Receiver $\rightarrow$ $\{B = g^b \bmod p, \text{Receiver\_Cert}, \sigma_R\}$ $\rightarrow$ Sender
    \item Sender verifies: $\text{hash}(\text{Receiver\_Cert}) \stackrel{?}{=} \sigma_R^{e_{CA}} \bmod n_{CA}$
    \item If valid: Sender trusts $B$ belongs to Receiver
    \item Both compute: $k_{SR} = g^{ab} \bmod p$ using \textbf{authenticated} keys
\end{enumerate}

\textbf{Step 3: MITM Attack Failure}
\begin{itemize}
    \item Sniffer intercepts messages containing certificates
    \item Sniffer \textcolor{red}{\textbf{cannot}} create valid certificate for itself (lacks $d_{CA}$)
    \item If Sniffer forwards own public key without certificate: rejected
    \item If Sniffer tries to modify certificate: signature verification fails
    \item Computing $d_{CA}$ from $(e_{CA}, n_{CA})$ requires factoring $n_{CA}$ (computationally infeasible)
    \item \textcolor{green!60!black}{\textbf{Result:}} Attack detected and prevented
\end{itemize}
\end{tcolorbox}

\vspace{0.3cm}

\subsubsection*{Security Foundation}

The security of this system relies on two hard mathematical problems:

\begin{itemize}[label={\color{maincolor}$\blacktriangleright$}]
    \item \textbf{Discrete Logarithm Problem (DLP):} Given $g$, $p$, and $g^x \bmod p$, computing $x$ is computationally hard (protects DH private keys)
    \item \textbf{Integer Factorization Problem:} Given $n = pq$, finding $p$ and $q$ is computationally hard (protects RSA private key)
\end{itemize}

\textbf{Key Insight:} Certificates transform the key exchange problem from \textit{``How do I securely get Bob's public key?''} to \textit{``How do I verify this public key belongs to Bob?''} The latter is solved by the CA's digital signature, which cryptographically binds identity to public key.


\section{System Architecture}

\begin{tcolorbox}[
    enhanced,
    colback=green!5!white,
    colframe=green!65!black,
    boxrule=1pt,
    arc=3mm,
    left=10pt, right=10pt, top=10pt, bottom=10pt
]
This OMNeT++ project consists of \textbf{15 interconnected modules} that work together to simulate a complete email transmission system with security features. Each module has a specific role in the email delivery pipeline, from composition to final retrieval.
\end{tcolorbox}

\vspace{0.5cm}

\subsection*{Network Topology}

\begin{figure}[H]
    \centering
    \includegraphics[width=0.95\textwidth]{topology.png}
    \caption{OMNeT++ Network Topology showing all 15 modules including sender, receiver, routers, DNS servers, mail transfer agents (MTA), spool, mailbox, Certificate Authority (CA), and the malicious sniffer positioned between mta\_Client\_SS and the router.}
    \label{fig:topology}
\end{figure}

\vspace{0.5cm}

\subsection{Core Communication Modules}

\begin{modulebox}{1. Sender \& 2. Receiver Modules}
\textbf{Files:} \code{sender.cc, receiver.cc}

Sender initiates email transmission by composing messages, querying DNS, and submitting via HTTP. Receiver retrieves emails using IMAP protocol through MAA\_Client.
\end{modulebox}

\vspace{0.5cm}

\subsection{Infrastructure Services}

\begin{modulebox}{3. DNS, 4. HTTP, 13. Router Modules}
\textbf{Files:} \code{dns.cc, http.cc, router.cc}

\textbf{DNS:} Resolves domain names to IP addresses for mail server lookup.

\textbf{HTTP:} Handles web-based email submission and retrieval operations.

\textbf{Router:} Routes packets throughout the network with flooding support for unknown destinations.
\end{modulebox}

\vspace{0.5cm}

\subsection{Mail Transfer Agents (MTAs)}

\begin{modulebox}{5-9. MTA Chain: Client\_S $\rightarrow$ Server\_S $\rightarrow$ Spool $\rightarrow$ Client\_SS $\rightarrow$ Server\_RS}
\textbf{Files:} \code{mta\_Client\_S.cc, mta\_Server\_S.cc, spool.cc, mta\_Client\_SS.cc, mta\_Server\_RS.cc}

Multi-stage email relay system implementing SMTP protocol:
\begin{itemize}
    \item \textbf{MTA\_Client\_S:} Receives submissions via HTTP, initiates SMTP
    \item \textbf{MTA\_Server\_S:} Accepts SMTP connections, forwards to spool
    \item \textbf{Spool:} Temporary FIFO queue for emails awaiting delivery
    \item \textbf{MTA\_Client\_SS:} Relay client with Diffie-Hellman \& certificate support
    \item \textbf{MTA\_Server\_RS:} Recipient's server with RSA encryption \& certificate validation
\end{itemize}
\textbf{Parameters:} \code{useCertificates} (default: false), \code{maxMessageSizeBytes} (2MB)
\end{modulebox}

\vspace{0.5cm}

\subsection{Mail Storage and Access}

\begin{modulebox}{10-12. Mailbox, MAA\_Server, MAA\_Client}
\textbf{Files:} \code{mailbox.cc, maa\_Server.cc, maa\_Client.cc}

\textbf{Mailbox:} Stores emails and sends new mail notifications.

\textbf{MAA\_Server:} IMAP server handling fetch requests from mailbox.

\textbf{MAA\_Client:} Client-side agent delivering emails to Receiver via IMAP.
\end{modulebox}

\vspace{0.5cm}

\subsection{Security Modules}

\begin{warningbox}[14. MaliciousSniffer - ATTACK COMPONENT]
\textbf{File:} \code{sniffer.cc}

Simulates MITM attacker with key substitution, dual session management, message modification, RSA factorization attempts, and traffic logging. Becomes passive when certificates detected.
\end{warningbox}

\vspace{0.3cm}

\begin{successbox}[15. CA (Certificate Authority) - DEFENSE]
\textbf{File:} \code{CA.cc}

Issues RSA-signed digital certificates to prevent MITM attacks. Provides cryptographic proof of identity that attackers cannot forge.

\textbf{Parameters:} \code{enabled} (false), \code{certificateValidityPeriod} (3600s)
\end{successbox}

\vspace{0.5cm}

\subsection{Scenario 1: MITM Attack Succeeds (No Certificates)}

\begin{warningbox}
\textbf{Configuration:} \texttt{Scenario1\_MITM\_Attack} in \texttt{omnetpp.ini}\\
\textbf{Parameters:} \texttt{**.useCertificates = false} and \texttt{**.ca.enabled = false}\\
\textbf{Purpose:} Demonstrate the vulnerability of unauthenticated key exchange
\end{warningbox}

\vspace{0.3cm}

\textbf{System Workflow:}

\vspace{0.3cm}
\textbf{Phase 1: Initialization}
\begin{itemize}[leftmargin=1cm]
    \item \texttt{Sender (mta\_Client\_SS)} at address 300 initializes without certificates
    \item \texttt{Receiver (mta\_Server\_RS)} at address 500 initializes without certificates
    \item \texttt{Sniffer} positioned between Sender and Router, ready to intercept
    \item No Certificate Authority involved (disabled in configuration)
\end{itemize}

\vspace{0.3cm}
\textbf{Phase 2: Compromised Key Exchange}
\begin{itemize}[leftmargin=1cm]
    \item \texttt{Sender} generates DH parameters (g=5, p=23, private key) and RSA keys (e, n)
    \item \texttt{Sender} $\rightarrow$ \textcolor{maincolor}{\textbf{DH\_HELLO}} (DH public=5432, RSA e=17, n=3233) $\rightarrow$ Router
    \item \textcolor{red}{\textbf{[SNIFFER INTERCEPTS]}} - Message never reaches Receiver
    \item \textcolor{red}{\textbf{Sniffer}} generates own DH parameters and RSA keys
    \item \textcolor{red}{\textbf{Sniffer}} $\rightarrow$ \textcolor{accentcolor}{\textbf{Forged DH\_HELLO}} (Sniffer's keys) $\rightarrow$ \texttt{Receiver}
    \item \texttt{Receiver} computes shared secret with Sniffer's DH public key (e.g., shared\_secret=8)
    \item \texttt{Receiver} $\rightarrow$ \textcolor{maincolor}{\textbf{DH\_HANDSHAKE}} (DH public, RSA keys) $\rightarrow$ Router
    \item \textcolor{red}{\textbf{[SNIFFER INTERCEPTS]}} - Message never reaches Sender
    \item \textcolor{red}{\textbf{Sniffer}} $\rightarrow$ \textcolor{accentcolor}{\textbf{Forged DH\_HANDSHAKE}} (Sniffer's keys) $\rightarrow$ \texttt{Sender}
    \item \texttt{Sender} computes shared secret with Sniffer's DH public key (e.g., shared\_secret=15)
\end{itemize}

\vspace{0.3cm}
\textbf{Phase 3: Two Separate Encrypted Sessions Established}
\begin{itemize}[leftmargin=1cm]
    \item \textbf{Session A:} \texttt{Sender} $\leftrightarrow$ \textcolor{red}{\textbf{Sniffer}} (shared secret = 15)
    \item \textbf{Session B:} \textcolor{red}{\textbf{Sniffer}} $\leftrightarrow$ \texttt{Receiver} (shared secret = 8)
    \item Both parties believe they have secure end-to-end encryption
    \item In reality: Sniffer controls both encryption sessions
\end{itemize}

\vspace{0.3cm}
\textbf{Phase 4: SMTP Communication - Email Transmission}
\begin{itemize}[leftmargin=1cm]
    \item \texttt{Sender} encrypts email using Session A keys (XOR with secret 15)
    \item \texttt{Sender} $\rightarrow$ \textcolor{maincolor}{\textbf{EHLO, MAIL FROM:} alice@example.com} (encrypted) $\rightarrow$ \textcolor{red}{\textbf{Sniffer}}
    \item \textcolor{red}{\textbf{Sniffer}} decrypts with Session A key (XOR with secret 15) - \textcolor{red}{\textbf{READS CONTENT}}
    \item \textcolor{red}{\textbf{Sniffer}} $\rightarrow$ \textcolor{maincolor}{\textbf{RCPT TO:} bob@example.com} (encrypted) $\rightarrow$ \textcolor{red}{\textbf{Sniffer}}
    \item \textcolor{red}{\textbf{Sniffer}} decrypts and \textcolor{red}{\textbf{LOGS: FROM, TO, SUBJECT, CONTENT}}
    \item \textcolor{red}{\textbf{Sniffer}} re-encrypts with Session B keys (XOR with secret 8)
    \item \textcolor{red}{\textbf{Sniffer}} $\rightarrow$ \textcolor{maincolor}{Re-encrypted SMTP messages} $\rightarrow$ \texttt{Receiver}
    \item \texttt{Receiver} successfully decrypts using Session B keys
    \item Email delivered normally - \textcolor{red}{\textbf{both parties unaware of compromise}}
\end{itemize}

\vspace{0.3cm}
\textbf{Attack Outcome:}
\begin{successbox}
\textcolor{red}{\textbf{★★★ MITM ATTACK SUCCESSFUL ★★★}}

\textbf{Console Output Shows:}
\begin{verbatim}
INTERCEPTED EMAIL:
  FROM: alice@example.com
  TO: bob@example.com
  SUBJECT: Test Email
  CONTENT: Hello from Sender
\end{verbatim}

\textbf{Impact:}
\begin{itemize}
    \item ✗ Complete confidentiality breach - All email content stolen
    \item ✗ No authentication - Sniffer impersonates both parties
    \item ✗ No detection - Neither Sender nor Receiver knows attack occurred
    \item ✗ Two separate encrypted sessions under attacker's control
\end{itemize}
\end{successbox}

\vspace{0.5cm}

\begin{figure}[H]
    \centering
    \includegraphics[width=0.85\textwidth]{Sniffer Decrypt Message .png}
    \caption{Scenario 1: Sniffer successfully decrypts and reads email content during MITM attack. The console shows intercepted FROM, TO, SUBJECT, and message CONTENT fields.}
    \label{fig:sniffer_decrypt}
\end{figure}

\vspace{0.3cm}

\begin{figure}[H]
    \centering
    \includegraphics[width=0.85\textwidth]{Email passing with encryption.png}
    \caption{Scenario 1: Email transmission with encryption but without certificate verification. Messages pass through the sniffer who decrypts, reads, and re-encrypts all SMTP communication.}
    \label{fig:email_encryption}
\end{figure}

\vspace{0.4cm}

\subsection{Scenario 2: Certificates Prevent MITM}

\begin{infobox}
\textbf{Configuration:} \texttt{Scenario2\_With\_Certificates} in \texttt{omnetpp.ini}\\
\textbf{Parameters:} \texttt{**.useCertificates = true}, \texttt{**.ca.enabled = true}, \texttt{**.ca.certificateValidityPeriod = 3600s}\\
\textbf{Purpose:} Demonstrate how PKI and digital certificates prevent MITM attacks
\end{infobox}

\vspace{0.3cm}

\textbf{System Workflow:}

\vspace{0.3cm}
\textbf{Phase 1: Certificate Authority Initialization}
\begin{itemize}[leftmargin=1cm]
    \item \texttt{CA (Certificate Authority)} at address 900 initializes
    \item CA generates RSA key pair: Public key (e=17, n=3233), Private key (d=2753)
    \item CA announces availability to the network
    \item Console displays: \texttt{CA INITIALIZED | CA Address: 900 | RSA Public Key: e=17, n=3233}
\end{itemize}

\vspace{0.3cm}
\textbf{Phase 2: Certificate Issuance}
\begin{itemize}[leftmargin=1cm]
    \item \texttt{Sender} $\rightarrow$ \textcolor{maincolor}{\textbf{CERT\_REQUEST}} (identity, address 300, RSA public, DH public) $\rightarrow$ \texttt{CA}
    \item \texttt{CA} creates certificate data string: \texttt{"Sender|300|17|3233|5432"}
    \item \texttt{CA} signs certificate using RSA private key (d=2753): \texttt{signature = signCertificate(certData)}
    \item \texttt{CA} $\rightarrow$ \textcolor{sectioncolor}{\textbf{CERT\_RESPONSE}} (certificate + signature, valid until 3600s) $\rightarrow$ \texttt{Sender}
    \item \texttt{Sender} stores certificate: \texttt{myCertificate = receivedCert}
    \item \texttt{Receiver} $\rightarrow$ \textcolor{maincolor}{\textbf{CERT\_REQUEST}} (identity, address 500, RSA public, DH public) $\rightarrow$ \texttt{CA}
    \item \texttt{CA} issues and signs Receiver's certificate
    \item \texttt{CA} $\rightarrow$ \textcolor{sectioncolor}{\textbf{CERT\_RESPONSE}} (certificate + signature, valid until 3600s) $\rightarrow$ \texttt{Receiver}
    \item \texttt{Receiver} stores certificate: \texttt{myCertificate = receivedCert}
\end{itemize}

\vspace{0.3cm}
\textbf{Phase 3: Authenticated Key Exchange with Certificate Validation}
\begin{itemize}[leftmargin=1cm]
    \item \texttt{Sender} $\rightarrow$ \textcolor{maincolor}{\textbf{DH\_HELLO}} + \textcolor{sectioncolor}{\textbf{Sender Certificate}} + \textcolor{sectioncolor}{\textbf{Signature}} $\rightarrow$ Router
    \item \textcolor{orange}{\textbf{[SNIFFER DETECTS CERTIFICATE]}}
    \item \textcolor{orange}{\textbf{Sniffer}} attempts to forge certificate but \textcolor{red}{\textbf{FAILS}} (no CA private key)
    \item \textcolor{orange}{\textbf{Sniffer}} console: \texttt{✗ Cannot forge valid certificate signature - Attack failed!}
    \item \textcolor{orange}{\textbf{Sniffer}} $\rightarrow$ \textcolor{maincolor}{\textbf{Forwards unmodified message}} $\rightarrow$ \texttt{Receiver}
    \item \texttt{Receiver} extracts certificate from DH\_HELLO
    \item \texttt{Receiver} verifies signature: \texttt{bool valid = verifyCertificate(cert, signature, CA\_e, CA\_n)}
    \item \texttt{Receiver} checks expiration: \texttt{if (currentTime < cert.expiryTime) → Valid}
    \item \texttt{Receiver} console: \texttt{✓ Signature valid (CA verification passed)}
    \item \texttt{Receiver} console: \texttt{✓ Certificate not expired}
    \item \texttt{Receiver} console: \texttt{✓ Identity confirmed: Sender}
    \item \texttt{Receiver} $\rightarrow$ \textcolor{maincolor}{\textbf{DH\_HANDSHAKE}} + \textcolor{sectioncolor}{\textbf{Receiver Certificate}} + \textcolor{sectioncolor}{\textbf{Signature}} $\rightarrow$ Router
    \item \textcolor{orange}{\textbf{Sniffer}} $\rightarrow$ \textcolor{maincolor}{\textbf{Forwards unmodified message}} $\rightarrow$ \texttt{Sender}
    \item \texttt{Sender} verifies Receiver's certificate using CA public key
    \item \texttt{Sender} console: \texttt{✓ Signature valid | ✓ Certificate not expired | ✓ Identity confirmed}
    \item Both parties compute shared DH secret using \textbf{authentic} public keys
\end{itemize}

\vspace{0.3cm}
\textbf{Phase 4: Secure End-to-End Communication}
\begin{itemize}[leftmargin=1cm]
    \item \texttt{Sender} and \texttt{Receiver} establish direct encrypted session (shared secret known only to them)
    \item \texttt{Sender} $\rightarrow$ \textcolor{maincolor}{\textbf{Encrypted SMTP: EHLO, MAIL, RCPT, DATA}} $\rightarrow$ \textcolor{orange}{\textbf{Sniffer}}
    \item \textcolor{orange}{\textbf{Sniffer}} \textcolor{red}{\textbf{cannot decrypt}} (lacks shared secret) $\rightarrow$ forwards encrypted messages
    \item \textcolor{orange}{\textbf{Sniffer}} $\rightarrow$ \textcolor{maincolor}{\textbf{Encrypted messages}} $\rightarrow$ \texttt{Receiver}
    \item \texttt{Receiver} decrypts successfully using authentic shared secret
    \item Email delivered securely - \textcolor{green!60!black}{\textbf{MITM attack prevented}}
\end{itemize}

\vspace{0.3cm}
\textbf{Defense Outcome:}
\begin{successbox}
\textcolor{green!60!black}{\textbf{�� SECURE COMMUNICATION ESTABLISHED!}}

\textbf{Console Output Shows:}
\begin{verbatim}
┌─────────────────────────────────────────────┐
│ Sniffer: Attempting MITM attack...         │
│   ✗ ATTACK FAILED!                         │
│   ✗ Cannot forge valid certificate         │
│   ✗ Receiver rejected fake certificate     │
│   ✗ Connection refused                     │
└─────────────────────────────────────────────┘
�� Secure communication established!
\end{verbatim}

\textbf{Protection Achieved:}
\begin{itemize}
    \item ✓ Authentication - Certificates bind public keys to verified identities
    \item ✓ Integrity - CA signature prevents certificate forgery
    \item ✓ Confidentiality - End-to-end encryption with authenticated keys
    \item ✓ Detection - Invalid certificates immediately rejected
    \item ✓ Trust Model - CA public key known to all parties (out-of-band)
\end{itemize}
\end{successbox}

\vspace{0.5cm}

\begin{figure}[H]
    \centering
    \includegraphics[width=0.9\textwidth]{attack fail.png}
    \caption{Scenario 2: MITM attack failure. Console output shows the sniffer's attempt to forge certificates failed. The warning message "⚠ CERTIFICATE IN DH\_HANDSHAKE - Attack Failed!" indicates certificate-based authentication successfully prevented the attack. Receiver uses certificate-based authentication and the sniffer cannot forge valid CA signatures.}
    \label{fig:attack_fail}
\end{figure}

\vspace{0.2cm}
\noindent\textit{Topology reference:} Modules and links are defined in `src/email.ned`. The sniffer is placed between `mta\_Client\_SS` and the router; the CA connects to the router and can be enabled/disabled per scenario.


\section*{Protocol Overview}

The simulation implements six protocols for complete email lifecycle:

\textbf{1. DNS:} Resolves domain names to IP addresses (\texttt{DNS\_QUERY}, \texttt{DNS\_RESPONSE})

\textbf{2. HTTP:} Submits emails from sender to MTA (\texttt{HTTP\_GET}, \texttt{HTTP\_RESPONSE})

\textbf{3. SMTP:} Transfers emails between mail servers using command sequence HELO/EHLO → MAIL FROM → RCPT TO → DATA → QUIT

\textbf{4. IMAP:} Retrieves emails from mailbox to receiver (\texttt{IMAP\_FETCH}, \texttt{IMAP\_RESPONSE})

\textbf{5. Notification:} Custom push notification when new mail arrives (\texttt{NOTIFY\_NEWMAIL})

\textbf{6. Push:} Internal reliable handshake for component forwarding (\texttt{PUSH\_REQUEST}, \texttt{PUSH\_ACK})

\textbf{Message Flow:} DNS resolution → HTTP submission → SMTP relay → Spool (Push) → Mailbox → Notification → IMAP retrieval


\section*{Discussion}

This project demonstrates that encryption without authentication is insufficient for secure communication. The MITM simulation revealed how attackers exploit Diffie-Hellman key exchange by substituting public keys and establishing dual encrypted sessions. Without certificates, both parties remain unaware of the compromise. However, certificate-based authentication immediately neutralizes such attacks—the sniffer cannot forge CA signatures and must operate transparently. This reinforces that security requires layered defenses: encryption for confidentiality, authentication for identity verification, and certificates for trust establishment. The OMNeT++ simulation bridges theory and practice, providing tangible insights into protocol vulnerabilities and defense mechanisms.

\section*{Conclusion}

This project conclusively demonstrates that robust digital communication security demands both strong cryptographic algorithms and rigorous authentication mechanisms. While encryption protects message content, certificates prevent impersonation by cryptographically binding identities to public keys. The simulation shows MITM attacks succeed when authentication is absent but fail when certificates are deployed. These findings emphasize that security best practices must integrate encryption, authentication, and trusted certificate authorities. The project serves as an educational tool demonstrating how protocol design decisions directly impact security effectiveness in real-world network scenarios.








\section[References]{\Huge\textbf{References}}
\vspace{0.5cm}

\begin{thebibliography}{10}

    \bibitem{omnetpp}
    A. Varga, "OMNeT++," in \textit{Modeling and Tools for Network Simulation}, K. Wehrle, M. Güneş, and J. Gross, Eds. Berlin, Heidelberg: Springer Berlin Heidelberg, 2010, pp. 35–59. \\
    Available: \href{https://omnetpp.org/}{https://omnetpp.org/}
    
    \bibitem{omnetpp_manual}
    "OMNeT++ Discrete Event Simulator - User Manual, Version 6.2," OMNeT++ Documentation, 2025. \\
    Available: \href{https://doc.omnetpp.org/omnetpp/manual/}{https://doc.omnetpp.org/omnetpp/manual/}
    

    
    
    
    \bibitem{smtp_protocol}
    J. Klensin, "Simple Mail Transfer Protocol," RFC 5321, Internet Engineering Task Force, Oct. 2008. \\
    Available: \href{https://www.rfc-editor.org/rfc/rfc5321}{https://www.rfc-editor.org/rfc/rfc5321}
    
    \bibitem{diffie_hellman}
    W. Diffie and M. Hellman, "New directions in cryptography," in \textit{IEEE Transactions on Information Theory}, vol. 22, no. 6, pp. 644-654, November 1976. \\
    DOI: \href{https://doi.org/10.1109/TIT.1976.1055638}{10.1109/TIT.1976.1055638}

    \bibitem{Presentation1}
    "SMTP Protocol - Network Security Presentation," Google Slides Presentation. \\
    Available: \href{https://docs.google.com/presentation/d/120bFVOPYJSHXJ41zwws-RYSdqnI8ezrY/edit?slide=id.p1#slide=id.p1}{https://docs.google.com/presentation/d/120bFVOPYJSHXJ41zwws-RYSdqnI8ezrY}
    
    \bibitem{Presentation2}
    "MITM Attack and Certificate-Based Defense," Google Slides Presentation. \\
    Available: \href{https://docs.google.com/presentation/d/1UAqh5GC-xQAV8R46ZLcln4d_lT-Wf0OC/edit?slide=id.g37ece01983c_3_0#slide=id.g37ece01983c_3_0}{https://docs.google.com/presentation/d/1UAqh5GC-xQAV8R46ZLcln4d\_lT-Wf0OC}
    
    
    \end{thebibliography}


\end{document}
